\documentclass{article}

\usepackage{hyperref}
\usepackage{graphicx}
\usepackage{multirow}
\usepackage{subfig}
\usepackage{float}
\usepackage{longtable}
\usepackage{blindtext}

\graphicspath{{./fig/}}

\begin{document}


\title{MEAM 620 - Robotics\\Project 3 - Visual Odometry}
\date{April 27, 2012}
\author{Jordan Brindza}
\maketitle
  

\section{Introduction}

  For this project I implemented an automous helicopter that can plan and
  traverse a trajectory through an known environment using a vision based
  localization. 

  For this phase of the project I used color regions as my main feature. First,
  the foreground of each image was segmented out (the walls and floor colors
  were known). Connected regions in the foreground image were found and any
  regions consisting of pixels of similar color were merged. Then the center of mass
  (based on pixel densities) for each region was computed. This was done both
  the left and right stereo images.

  \begin{figure}[H]
    \begin{center}$
      \begin{tabular}{cc}
        \includegraphics[width=0.5\columnwidth]{block_match_left_rgb}
        &
        \includegraphics[width=0.5\columnwidth]{block_match_left_block_seg} 
      \end{tabular}$
    \end{center}
    \caption{Example RGB image (left) and segmented regions (right).}
  \end{figure}

  Correspondence was then found between the left and right image regions by
  matching similar colors and ensuring that their centroids lie close to their
  epipolar line. The disparity of the block is computed as the average disparity
  for the entire color color region that is found using the OpenCV SSD disparity
  algorithm. If the disparity for the block is not available then the difference
  between the region centroids for the left and right image is used. Finally, the 
  true world positions of the blocks are found by matching the block color from 
  the image with the blocks in the map and the relative transform between the
  observed positions and the actual positions is computed using a least squares
  fit to find the pose estimate for the robot in the world. ~\cite{arun}

  \begin{figure}[H]
    \begin{center}
      \includegraphics[width=0.8\columnwidth]{block_match_stereo_corr}
    \end{center} 
    \caption{Example stereo correspondence.}
  \end{figure}

  The visual pose estimate is computed at 10 Hz and then fused with the
  quadrotor pose estimate found from integrating the noisy linear and angular
  velocity estimates provided by the simulator. A low pass, linear filter is
  used to combine the visual estimates. 

  The only changes that were made to the controller from project 1 were lowering
  some gains to compensated for the slower, 100 Hz, update rate of the
  controller and to make the desired trajectory with the yaw pointing in the
  direction of travel.

\section{Results}
  
  The simulated quadrotor was able successfully complete environments 1 and 7
  (both the original version and the tweaked one, 7.2, with more features on the
  center columns). Below is a table containing the total time it took to
  complete the course and the maximum error in position and orientation along
  the trajectory. There are also plots of the final trajectory of the quadrotor
  for each environment with a comparison to the ground truth data from the
  simulator.

  There is data from two trials of environment 1, one trial of the original
  environment 7 and one of the new environment 7 (indicated as environment
  7.2).

  A video of the quadrotor completing the environments can be viewed here:
  \url{http://youtu.be/PAapLIL_a-E}

\begin{center}
  \begin{tabular}{|c|c|c|c|}
    \hline
    Environment & Completion Time [s] & Max Position Error [m] & Max Orientation Error \\
    \hline
    1         & $85.40 $ &  $0.5220$ & $1.9745$ \\
    1         & $85.72 $ &  $0.3895$ & $0.1701$ \\
    7         & $120.52$ &  $0.9430$ & $0.5883$ \\
    7.2       & $120.82$ &  $1.2152$ & $1.9514$ \\
    \hline
  \end{tabular}
  \label{tab:myfirsttable}
\end{center}



%  \begin{figure}[H]
%    \begin{center}$
%      \begin{tabular}{cc}
%        \includegraphics[width=0.5\columnwidth]{auto_E_001}
%        &
%        \includegraphics[width=0.5\columnwidth]{man_E_001} \\
%      \end{tabular}$
%    \end{center}
%    \caption{Automatic (left) and manual (right) correspondence for stereo pair 1.}
%  \end{figure}

  \begin{figure}[H]
    \begin{center}
      \includegraphics[width=1.0\columnwidth]{env1_final_paths}
    \end{center} 
    \caption{Final trajectory for environment 1.}
  \end{figure}

  \begin{figure}[H]
    \begin{center}
      \includegraphics[width=1.0\columnwidth]{env1_pos}
    \end{center} 
    \caption{Plot of the quadrotor cartesian position for environment 1.}
  \end{figure}

  \begin{figure}[H]
    \begin{center}
      \includegraphics[width=1.0\columnwidth]{env1_rpy}
    \end{center} 
    \caption{Plot of the quadrotor orientation (roll, pitch and yaw) for environment 1.}
  \end{figure}

  \begin{figure}[H]
    \begin{center}
      \includegraphics[width=1.0\columnwidth]{env1_try2_final_paths}
    \end{center} 
    \caption{Final trajectory for environment 1 (second run).}
  \end{figure}

  \begin{figure}[H]
    \begin{center}
      \includegraphics[width=1.0\columnwidth]{env1_try2_pos}
    \end{center} 
    \caption{Plot of the quadrotor cartesian position for environment 1 (second run).}
  \end{figure}

  \begin{figure}[H]
    \begin{center}
      \includegraphics[width=1.0\columnwidth]{env1_try2_rpy}
    \end{center} 
    \caption{Plot of the quadrotor orientation (roll, pitch and yaw) for environment 1 (second run).}
  \end{figure}


  \begin{figure}[H]
    \begin{center}
      \includegraphics[width=1.0\columnwidth]{env7_final_paths}
    \end{center} 
    \caption{Final trajectory for environment 7.}
  \end{figure}

  \begin{figure}[H]
    \begin{center}
      \includegraphics[width=1.0\columnwidth]{env7_pos}
    \end{center} 
    \caption{Plot of the quadrotor cartesian position for environment 7.}
  \end{figure}

  \begin{figure}[H]
    \begin{center}
      \includegraphics[width=1.0\columnwidth]{env7_rpy}
    \end{center} 
    \caption{Plot of the quadrotor orientation (roll, pitch and yaw) for environment 7.}
  \end{figure}


  \begin{figure}[H]
    \begin{center}
      \includegraphics[width=1.0\columnwidth]{env7_2_final_paths}
    \end{center} 
    \caption{Final trajectory for environment 7.2 (after adding more features to center columns).}
  \end{figure}

  \begin{figure}[H]
    \begin{center}
      \includegraphics[width=1.0\columnwidth]{env7_2_pos}
    \end{center} 
    \caption{Plot of the quadrotor cartesian position for environment 7.2 (after adding more features to center columns).}
  \end{figure}

  \begin{figure}[H]
    \begin{center}
      \includegraphics[width=1.0\columnwidth]{env7_2_rpy}
    \end{center} 
    \caption{Plot of the quadrotor orientation (roll, pitch and yaw) for environment 7.2 (after adding more features to center columns).}
  \end{figure}

\bibliography{report} \bibliographystyle{plain}

\end{document}

